\chapter{Сферы применения технологий Re-Id}
\label{ch:applications}

Обсудим основные сферы применения технологий \reid. Как упоминалось выше, большинство методов способны обрабатывать данные совершенно разных предметных областей, или \textit{доменов}. Действительно, основные техники, применяемые в задаче Re-Id, такие как метрическое обучение, оптимизация ранжирования, использование дополнительной информации, аугментации, работа с видео-последовательностью, могут быть распространены и на другие домены. В то же время часть методов, таких как детекция частей тела, нарезка изображения на части, обучение локальных признаков, либо не могут быть перенесены на данные другой предметной области, либо должны быть преобразованы. Таким образом, различные сферы применения Re-Id технологий, перечисленные ниже, объединены общей постановкой задачи и структурой работы системы, однако каждая из них содержит свои нюансы.

\section{Распознавание по лицу для контроля доступа}

В современном мире широкую популярность имеют технологии \textit{Face-ID}, позволяющие производить идентификацию по фотографии лица человека. Во-первых, они используются для аутентификации пользователя в компьютерах, смартфонах и других устройствах. Во-вторых, те же технологии применяются для работы систем оплаты \textit{face-pay} в транспорте и магазинах. Кроме этого, техники идентификации по лицу могут быть встроены в сложные многоуровневые системы доступа к системам повышенной безопасности.


\section{Идентификация людей для обеспечения безопасности}

В данном случае речь идет об обнаружении подозрительных личностей, или же конкретных людей, которые могут быть вовлечены в противоправную деятельность. Данный вариант применения технологии наиболее акутален для предприятий, в том числе секретных, а также для общественных учреждений с высоким скоплением людей. Также к этой сфере применения относится отслеживание перемещения подозреваемого человека между разными локациями, в которых установлены камеры видеонаблюдения. Технологии ре-идентификации позволяют строить траекторию движения конкретного человека, исходя из видео-данных с этих камер.


\section{Отслеживание транспортных средств}

Еще одной важной сферой применения технологий \reid\ является мониторинг движения автомобилей. Одной из задач в этой области является построение траектории движения конкретного автомобиля на дороге для контроля соблюдения ПДД: соблюдение скоростного режима, разметки полос, правил обгона и др. В таком случае методы Re-ID сочетаются с другими техниками компьютерного зрения. Во-первых, для построения траектории движения применяются в том числе техники \textit{трекинга}, то есть объединения информации о положении объекта на различных последовательных кадрах, основанных на последовательном вычислении и обновлении значений его скорости и направления в пространстве. Однако для улучшения построения траекторий, в частности, в условиях пересечения траекторий нескольких разных объектов, применимы в том числе технологии ре-идентификации, позволяющие правильно совместить продолжения траекторий объектов после их расхождения. Во-вторых, на этапе ре-идентификации используется также дополнительная информация о государственных номерах автомобилей, получаемая смежными методами компьютерного зрения. Таким образом, данный домен данных содержит общие черты с задачей \reid, а также некоторые уникальные особенности.


\section{Розничная торговля}

Здесь речь идет о нескольких путях применения ре-идентификации. Так, они могут быть использованы для анализа перемещений клиентов на территории магазина или торгового центра с помощью сопоставления их изображений на данных разных камер и построения общей траектории.

Кроме того, методы Re-Id применяются также для оптимизации контроля наличия товаров на прилавках. В данном случае задача сочетается с базовой постановкой классификации в компьютерном зрении, но не сводится к ней. Действительно, для корректной работы такой системы необходимо обрабатывать в том числе те объекты, для которых в данных имеется мало обучающих примеров, а также неизвестные объекты, в случае изменения ассортимента. Кроме того, в этой сфере применимы также технологии использования вспомогательной информации, такие как, например, наименования товаров или их текстовые описания.


\section{Поиск пропавших домашних животных}

Одной из основных сфер применения ре-идентификации, рассматриваемых в данной работе, является система поиска пропавших собак по данным городских камер видеонаблюдения. Данная система реализована в процессе выполнения бакалаврской работы. Рассмотрим подробно устройство этой системы в следующем разделе.



\endinput