\chapter{Заключение}
\label{ch:conclusion}

Таким образом, в работе рассмотрена задача задача \reid, ее научная теория и практические аспекты. Приведена математическая постановка задачи; проведен обширный анализ методов ее решения, а также областей их применимости. Рассмотрено реализованное автором практическое приложение технологий Re-Id. Выявлены пути развития применяемых методов и мотивация к использованию данных путей. Приведен анализ соответствующего базового решения адресованной проблемы, предложены его модификации. Проведены эксперименты по внедрению данных модификаций и анализ их результатов.

Так, ключевую часть решения задачи \reid\ составляют техники метрического обучения. При этом они могут быть дополнены и улучшены с помощью внесения дополнительных знаний о предметной области. Об этом свидетельствуют примеры из практических приложений, а также анализ лидирующих методов.

С другой стороны, улучшение качества с помощью совмещения нескольких техник является сложной задачей. Многие пути ее решения не приводят к успеху. Для построения качественной системы нужна тщательная подготовка данных и построение правильного процесса обучения.