% Аннотация
\renewenvironment{abstract}
{
  \centerline
  {\huge \textbf{Аннотация}}
  \begin{quote}
}
{
  \end{quote}
}

% Ключевые слова
\def\keywordname{{\bfseries \emph{Ключевые слова:}}}%
\def\keywords#1{\par\addvspace\medskipamount{\rightskip=0pt plus1cm
\def\and{\ifhmode\unskip\nobreak\fi\ $\cdot$
}\noindent\keywordname\enspace\ignorespaces#1\par}}

\begin{abstract}
В работе рассматривается задача компьютерного зрения Person Re-Identification. Она состоит в том, чтобы находить и распознавать целевые объекты по данным с камер видеонаблюдения. В качестве целевых объектов могут выступать как люди, так и объекты других типов $-$ автомобили, продукты на прилавках, животные. Одной из основных рассматриваемых сфер применения является реализованная в рамках проекта компании ООО "НКБТех"\ и в процессе выполнения бакалаврской работы система поиска пропавших собак по данным городских камер видеонаблюдения. Задача Person Re-Identification эффективно решается методами глубокого обучения, которые позволяют обрабатывать изображения с помощью нейронных сетей и техник метрического обучения. При этом качество классических подходов может быть улучшено с помощью подходов, использующих дополнительную информации. Мы исследуем, какие из этих подходов приводят к росту качества решения задачи.

    \keywords{Person Re-Identification, метрическое обучение, дополнительаня информация, мультимодальные модели, атрибуты, ключевые точки}
    
\end{abstract}
